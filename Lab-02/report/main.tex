\documentclass[12pt]{article}

\usepackage{amsmath}
\usepackage{amsfonts}
\usepackage{float}
\usepackage{fancyhdr}
\usepackage{graphicx}
\usepackage[colorlinks=true,linkcolor=blue, citecolor=red]{hyperref}
\usepackage{url}
\usepackage[top=.75in, left=.5in, right=.5in, bottom=1in]{geometry}
\usepackage[utf8]{vietnam}
\setlength{\headheight}{29.43912pt}

\pagestyle{fancy}
\lhead{
Báo cáo Lab 02 - Naive Bayes Classifier
}
\rhead{
Trường Đại học Khoa học Tự nhiên - ĐHQG HCM\\
CSC150008 - Xử lý ngôn ngữ tự nhiên ứng dụng
}
\lfoot{\LaTeX\ by \href{https://github.com/trhgquan}{Quan, Tran Hoang}}

\begin{document}
\noindent Sinh viên thực hiện: Trần Hoàng Quân (MSSV: 19120338)
\section{Chạy chương trình}
File thực thi là \texttt{main.py}. Thầy chạy \texttt{pip install -r requirements.txt} trước, sau đó chạy \texttt{python main.py} là được.
\section{Tiền xử lí \& huấn luyện mô hình}
Dữ liệu văn bản trải qua các bước tiền xử lí sau:
\begin{itemize}
\item Chuyển thành chữ thường.
\item Loại bỏ dấu câu.
\item Loại bỏ chữ số có trong câu.
\end{itemize}
Vì có sự chênh lệch lớn giữa số lượng câu có nhãn 2 với các nhãn còn lại, nên em chỉ chọn mỗi nhãn 50 câu để huấn luyện.

\section{Kết quả huấn luyện}
Test trên 500 câu trong tập tin test cho kết quả như sau:
\begin{table}[H]
\centering
\begin{tabular}{|c|c|c|c|}
\hline
label & P (precision) & R (recall) & f1-score \\
\hline
0 & 0.67 & 0.89 & 0.77 \\
1 & 0.50 & 0.44 & 0.47 \\
2 & 0.33 & 0.33 & 0.33 \\
3 & 0.40 & 0.43 & 0.41 \\
4 & 0.46 & 0.47 & 0.46 \\
5 & 0.64 & 0.42 & 0.51 \\
\hline
Tổng & 0.56 & 0.56 & 0.55 \\
\hline
\end{tabular}
\caption{Kết quả huấn luyện}
\end{table}
\end{document}